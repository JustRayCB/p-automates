\documentclass[a4paper, 12pt]{extarticle}

\input{preamble.tex} 


% Based on 'Fun Template 1', available at https://www.overleaf.com/latex/templates/fun-template-1/drwvdzsrpgzz


\begin{document}

%================= Settings front page =================
\titre{Rapport Informatique Fondamentale} %Titre du fichier .pdf
\UE{INFO-F302} %Nom de la UE
\sujet{Informatique Fondamentale} %Nom du sujet

\enseignant{E. \textsc{Filliot} \\ R. \textsc{Petit}}  %Nom des enseignants
% Use \\ TO BREAK LINE

\eleves{Hugo \textsc{Callens} \\ Rayan \textsc{Contuliano Bravo} \\ Ethan \textsc{Rogge}}
% \maketitle
\makemargins %Afficher les marges
\makefrontpage
% \maketoc

%=======================================================

% this is the orginal latex code of the template
% \input{example}

%================= Content =============================

\section{Modelisation d'automates en FNC} % (fold)
\label{sec:modelisation_d_automates_en_fnc_}

Afin de modéliser un automate en FNC, il faut tout d'abord définir les variables qui seront utilisées. 
Pour cela, nous avons décidé de créer une variable par état de l'automate, et une variable par transition. 
Ainsi, pour un automate à $n$ états et $m$ transitions, nous aurons $n+m$ variables. 

\subsection{Variables} % (fold)
\label{sub:variables}

\subsubsection{Variables d'états} % (fold) 
\label{ssub:variables_d_etats} 

Les variables d'états sont des variables booléennes qui représentent l'état dans lequel se trouve l'automate. 
Ainsi, pour un automate à $n$ états, nous aurons $n$ variables d'états. 
Ces variables sont nommées $S_i$ avec $i \in \{1, \dots, n\}$. 
La variable $S_i$ est vraie si et seulement si l'automate est dans l'état $i$.

% subsection Variables (end)



% section Modelisation d'automates en FNC (end)










%================= Bibliography ========================
% \newpage
% \phantomsection % Required if hyperref is used
% \addcontentsline{toc}{section}{References} % Adding bibliography to table of contents
% \printbibliography % Print the bibliography

\end{document}
