\documentclass[a4paper, 12pt]{extarticle}

\input{preamble.tex} 


% Based on 'Fun Template 1', available at https://www.overleaf.com/latex/templates/fun-template-1/drwvdzsrpgzz


\begin{document}

%================= Settings front page =================
\titre{Rapport Informatique Fondamentale} %Titre du fichier .pdf
\UE{INFO-F302} %Nom de la UE
\sujet{Informatique Fondamentale} %Nom du sujet

\enseignant{E. \textsc{Filliot} \\ R. \textsc{Petit}}  %Nom des enseignants
% Use \\ TO BREAK LINE

\eleves{Hugo \textsc{Callens} \\ Rayan \textsc{Contuliano Bravo} \\ Ethan \textsc{Rogge}}
% \maketitle
\makemargins %Afficher les marges
\makefrontpage
% \maketoc

%=======================================================

% this is the orginal latex code of the template
% \input{example}

%================= Content =============================

\section{Modelisation d'automates en FNC} % (fold)
\label{sec:modelisation_d_automates_en_fnc_}

% Sachant qu'il existe les ensemble $P$ et $N$ qui représentent respectivement 
% les états acceptants et non-acceptants d'un automate, nous avons décidé de 
% modéliser un automate en FNC de la manière suivante :

Voici quelques notations que nous utiliserons dans la suite de ce rapport : 
\begin{itemize}[label=$\bullet$]
    \item $P$ représente l'ensemble des mots acceptés par l'automate 
    \item $N$ représente l'ensemble des mots non-acceptés par l'automate 
    \item $\Sigma$ représente l'alphabet de l'automate 
    \item $k$ représente le nombre au plus d'états de l'automate
\end{itemize}

Afin de modéliser un automate en FNC, il faut tout d'abord définir les variables qui seront utilisées. 
Pour cela, nous avons décidé de créer une variable par état de l'automate, et une variable par transition. 
Ainsi, pour un automate à $n$ états et $m$ transitions, nous aurons $n+m$ variables. 

\subsection{Choix des variables} % (fold)
\label{sub:choix_des_variables}

\subsubsection{Etats} % (fold)
\label{sec:etats}

Nous définissons notre ensemble $Q = \{q_0, q_1, \dots, q_n | n \le k\}$, qui 
représente l'ensemble des états de l'automate. 
Nous pouvons définir les variables $e_i$ comme suit : 
\begin{itemize}[label=$\bullet$]
    \item $q_i = 1$ si l'état $i$ est acceptant 
    \item $q_i = 0$ si l'état $i$ est non-acceptant 
\end{itemize}

\subsubsection{Etats initial} % (fold)
\label{sec:etats_initial}

Nous définissons notre ensemble $I = \{i_0, i_1, \dots, i_n | n \le k\}$, qui 
représente l'ensemble des états initiaux de l'automate. 
Nous pouvons définir les variables $i_i$ comme suit : 
\begin{itemize}[label=$\bullet$]
    \item $i_i = 1$ si l'état $i$ est initial 
    \item $i_i = 0$ si l'état $i$ n'est pas initial 
\end{itemize}

% subsubsection Etats initial (end)

% subsubsection Etats (end)

\subsubsection{Transitions} % (fold) 
\label{sec:transitions} 


Nous définissons notre ensemble $\delta = \{d_{i, j, s_i} | i, j \in Q, s_i \in \Sigma\}$, qui 
représente l'ensemble des transitions de l'automate. 
Nous pouvons définir les variables $d_{i, j, s_i}$ comme suit : 
\begin{itemize}[label=$\bullet$]
    \item $d_{i, j, s_i} = 1$ si la transition avec le symbole $s_i$ existe entre l'état $i$ et l'état $j$ 
    \item $d_{i, j, s_i} = 0$ si la transition avec le symbole $s_i$ n'existe pas entre l'état $i$ et l'état $j$ 
\end{itemize} 

% subsubsection Transitions (end) 

\subsubsection{Exemples positifs} % (fold)
\label{sec:exemples_positifs}

Nous définissons $X_{p, i}$ comme étant la variable qui représente le fait que le mot $p$ est accepté par l'automate à l'état $q_i$.
Nous pouvons définir les variables $X_{p, i}$ comme suit : 
\begin{itemize}[label=$\bullet$]
    \item $X_{p, i} = 1$ si le mot $p$ est accepté par l'automate à l'état $q_i$ 
    \item $X_{p, i} = 0$ si le mot $p$ n'est pas accepté par l'automate à l'état $q_i$ 
\end{itemize}

% subsubsection Exemples positifs (end)

\subsubsection{Exemples négatifs} % (fold) 
\label{sec:exemples_negatifs} 

Nous définissons $Y_{n, i}$ comme étant la variable qui représente le fait que le mot $n$ n'est pas accepté par l'automate à l'état $q_i$. 

Nous pouvons définir les variables $Y_{n, i}$ comme suit : 
\begin{itemize}[label=$\bullet$]
    \item $Y_{n, i} = 1$ si le mot $n$ n'est pas accepté par l'automate à l'état $q_i$ 
    \item $Y_{n, i} = 0$ si le mot $n$ est accepté par l'automate à l'état $q_i$ 
\end{itemize}


\subsection{Contraintes} % (fold)
\label{sub:contraintes}

\begin{itemize}
    \item Il faut qu'il y ait un et un seul état initial 
    \item Il faut qu'il y ait au moins un état final
    \item Il faut que l'ensemble $P$ soit inclus dans l'ensemble des mots acceptés par l'automate 
    \item Un mot se trouvant dans l'ensemble $N$ ne peut pas être un état acceptant
    \item Chaque état doit avoir toutes les transitions possibles. 
\end{itemize}
% subsection Contraintes (end)

\subsection{FNC} % (fold) 
\label{sub:fnc} 

Voici les contraintes que nous avons définies pour modéliser un automate en FNC : 
\begin{itemize}[label=$\bullet$]
    \item Il faut qu'il y ait un et un seul état initial :
    \begin{itemize}[label=$\circ$]
        \item Un
            \begin{itemize}[label=$\diamond$]
                \item $\bigwedge_{i \in \{1, \dots, n\}} i_i$
            \end{itemize}
        \item et un seul
            \begin{itemize}[label=$\diamond$]
                \item $\bigwedge_{i, j \in \{1, \dots, n\} \wedge i \neq j}  \neg i_i \vee \neg i_j$
            \end{itemize} 

    \end{itemize}
    \item Il faut qu'il y ait au moins un état final : 
    \begin{itemize}[label=$\circ$]
        \item $\bigvee_{i \in \{1, \dots, n\}} q_i$
    \end{itemize}
    \item Il faut que l'ensemble $P$ soit inclus dans l'ensemble des mots acceptés par l'automate : 
    \begin{itemize}[label=$\circ$]
        \item $\bigwedge_{p \in P} \bigwedge_{i \in Q} X_{p, i}$
    \end{itemize}
    \item Un mot se trouvant dans l'ensemble $N$ ne peut pas être un état acceptant : 
    \begin{itemize}[label=$\circ$]
        \item $\bigwedge_{n\in N}\bigwedge_{i\in Q}(Y_{n,i})$
    \end{itemize} 
    \item Chaque état doit avoir toutes les transitions possibles (automate complet) : 
    \begin{itemize}[label=$\circ$]
        \item $\bigwedge_{i \in Q} \bigwedge_{j \in Q \setminus \{i\}} \bigwedge_{s_i \in \Sigma} \bigvee_{s_j \in \Sigma} d_{i, j, s_i}$ 
    \end{itemize} 
\end{itemize} 






% subsection Choix des variables (end)



% section Modelisation d'automates en FNC (end)










%================= Bibliography ========================
% \newpage
% \phantomsection % Required if hyperref is used
% \addcontentsline{toc}{section}{References} % Adding bibliography to table of contents
% \printbibliography % Print the bibliography

\end{document}
