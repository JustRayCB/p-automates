\documentclass[a4paper, 12pt]{extarticle}

\input{preamble.tex} 


% Based on 'Fun Template 1', available at https://www.overleaf.com/latex/templates/fun-template-1/drwvdzsrpgzz


\begin{document}

%================= Settings front page =================
\titre{Rapport Informatique Fondamentale} %Titre du fichier .pdf
\UE{INFO-F302} %Nom de la UE
\sujet{Informatique Fondamentale} %Nom du sujet

\enseignant{E. \textsc{Filliot} \\ R. \textsc{Petit}}  %Nom des enseignants
% Use \\ TO BREAK LINE

\eleves{Hugo \textsc{Callens} \\ Rayan \textsc{Contuliano Bravo} \\ Ethan \textsc{Rogge}}
% \maketitle
\makemargins %Afficher les marges
\makefrontpage
% \maketoc

%=======================================================

% this is the orginal latex code of the template
% \input{example}

%================= Content =============================

\section{Modelisation d'automates en FNC} % (fold)
\label{sec:modelisation_d_automates_en_fnc_}

% Sachant qu'il existe les ensemble $P$ et $N$ qui représentent respectivement 
% les états acceptants et non-acceptants d'un automate, nous avons décidé de 
% modéliser un automate en FNC de la manière suivante :

Voici quelques notations que nous utiliserons dans la suite de ce rapport : 
\begin{itemize}[label=$\bullet$]
    \item $P$ représente l'ensemble des mots acceptés par l'automate 
    \item $N$ représente l'ensemble des mots non-acceptés par l'automate 
    \item $\Sigma$ représente l'alphabet de l'automate 
    \item $k$ représente le nombre au plus d'états de l'automate
\end{itemize}

\noindent Afin de modéliser un automate en FNC, il faut tout d'abord définir les variables qui seront 
utilisées.Pour cela, nous avons décidé de créer une variable par état de l'automate, et une 
variable par transition. Ainsi, pour un automate à $n$ états et $m$ transitions, nous aurons 
$n+m$ variables. 

\subsection{Choix des variables} % (fold)
\label{sub:choix_des_variables}

\subsubsection{Etats} % (fold)
\label{sec:etats}

Nous définissons notre ensemble $Q = \{q_{0a}, q_{0na}, \dots, q_{la}, q_{lna} | l \le k\}$, il y a
donc $2k$ états dans $Q$ pour un automate à $k$ états. Nous pouvons définir les variables $q_i$ comme suit : 
\begin{itemize}[label=$\bullet$]
    \item $q_{ia} = 1$ si l'état $i$ est acceptant et donc $q_{ina} = 0$.
    \item $q_{ina} = 1$ si l'état $i$ est non-acceptant et donc $q_{ia} = 0$.
\end{itemize}

\subsubsection{Transitions} % (fold) 
\label{sec:transitions} 

Nous définissons notre ensemble $\delta = \{d_{i, l, j} | i, j \in Q, l \in \Sigma\}$, qui 
représente l'ensemble des transitions de l'automate. 
Nous pouvons définir les variables $d_{i, j, s_i}$ comme suit : 
\begin{itemize}[label=$\bullet$]
    \item $d_{i, j, l} = 1$ si la transition avec la lettre $l$ existe entre l'état $i$ et l'état $j$.
    \item $d_{i, j, l} = 0$ si la transition avec la lettre $l$ n'existe pas entre l'état $i$ et l'état $j$.
\end{itemize} 

% subsubsection Transitions (end) 

\subsubsection{Exécutions}
\label{sec:executions}

Nous définissons notre ensemble $E = \{e_{m, i, t} | i \in Q, t \in \{-1,\dots,\text{len}(m)\}\, m \in P \cup N\}$,
qui représente l'ensemble des exécutions de l'automate. Nous pouvons définir les variables $e_{m, i, t}$ comme suit :
\begin{itemize}[label=$\bullet$]
    \item $e_{m, i, t} = 1$ si l'automate est dans l'état $i$ après avoir lu les $t$ premières lettres du mot $m$.
    \item $e_{m, i, t} = 0$ si l'automate n'est pas dans l'état $i$ après avoir lu les $t$ premières lettres du mot $m$.
\end{itemize}


\subsection{Contraintes} % (fold)
\label{sub:contraintes}

\begin{enumerate}
    \item Il y a un unique état initial :
    \begin{equation*}
        q_{0a} \vee q_{0na}
    \end{equation*}
    \item Un état est exclusivement acceptant ou non-acceptant :
    \begin{equation*}
        \bigwedge_{\substack{q\in Q \\ i\in \{0,\dots,k\}}} \neg q_{ia} \vee \neg q_{ina}
    \end{equation*}
    \item Chaque état est acceptant ou non acceptant :
    \begin{equation*}
        \bigvee_{\substack{q\in Q \\ i\in \{0,\dots,k\}}} q_{ia} \vee q_{ina}
    \end{equation*}
    \item Chaque état a au plus une transition par lettre de l'alphabet :
    \begin{equation*}
        \bigwedge_{\substack{l\in \Sigma\\i,j,q \in Q}}\neg d_{i,j,l} \vee \neg d_{i,q,l}
    \end{equation*}
    \item S'il y a une transition pour la lettre $l$ d'un état $i$ à un état $j$, alors 
    $i$ et $j$ sont des états existants :
    \begin{equation*}
        \bigwedge_{\substack{i,j \in Q\\l \in \Sigma}} (\neg d_{i,j,l} \vee (q_{ia}\vee q_{ina})) \wedge (\neg d_{i,j,l} \vee (q_{ja}\vee q_{jna}))
    \end{equation*}
    \item Un état $i$ est acceptant s'il existe une exécution d'un mot $m$ de $P$ qui se termine sur l'état $i$ à l'étape $t=len(m)-1$ :
    \begin{equation*}
        \bigwedge_{\substack{t=len(m)-1\\q \in Q\\ m \in P\\ i \in \{0,\dots,k\}}} \neg e_{m,i,t} \vee q_{ia}
    \end{equation*}
    \item Si on a une exécution pour le mot $m$ à l'étape $t$ sur l'état $i$ et une transition de $i$ vers $j$ pour la lettre l,
    alors, on a une exécution pour le mot $m$ à l'étape $t+1$ sur l'état $j$ :
    \begin{equation*}
        \bigwedge_{\substack{i,j \in Q\\l \in \Sigma\\m \in P\\t \in \{0\dots,len(m)-1\}\\l=m[t]}} (\neg e_{m,i,t} \vee \neg d_{i,j,l} \vee e_{m,j,t+1})
    \end{equation*}
    \item Si on a une exécution pour le mot $m$ à l'étape $t$ sur l'état $i$ et une exécution pour le même mot $m$ à l'étape $t+1$ sur l'état $j$,
    alors, il existe une transition de $i$ vers $j$ pour la lettre $l$ :
    \begin{equation*}
        \bigwedge_{\substack{i,j \in Q\\l \in \Sigma\\m \in P\\t \in \{0,\dots,len(m)-1\}\\l=m[t+1]}} (\neg e_{m,i,t} \vee \neg e_{m,j,t+1} \vee d_{i,j,l})
    \end{equation*}
    \item Toutes les exécutions sur les mots de N doivent se terminer sur un état non-acceptant :
    \begin{equation*}
        \bigwedge_{\substack{m \in N\\i \in \{0,\dots,k\}\\t= len(m)-1}\\q\in Q} \neg e_{m,i,t} \vee q_{ina}
    \end{equation*}
    \item Toutes les exécutions sur les mots de P doivent exister :
    \begin{equation*}
        \bigwedge{\substack{m\in P}}\bigvee_{\substack{t\in\{0,\dots,len(word)-1\}\\i\in \{0,\dots,k\}}} e_{m,i,t}
    \end{equation*}
    \item Toutes les exécutions doivent commencer à l'état initial :
    \begin{equation*}
        \bigwedge_{\substack{m\in P\cup N}} e_{m,0,0}
    \end{equation*}
\end{enumerate}





%================= Bibliography ========================
% \newpage
% \phantomsection % Required if hyperref is used
% \addcontentsline{toc}{section}{References} % Adding bibliography to table of contents
% \printbibliography % Print the bibliography

\end{document}
