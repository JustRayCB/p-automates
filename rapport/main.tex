\documentclass[a4paper, 12pt]{extarticle}

\input{preamble.tex} 


% Based on 'Fun Template 1', available at https://www.overleaf.com/latex/templates/fun-template-1/drwvdzsrpgzz


\begin{document}

%================= Settings front page =================
\titre{Rapport Informatique Fondamentale} %Titre du fichier .pdf
\UE{INFO-F302} %Nom de la UE
\sujet{Informatique Fondamentale} %Nom du sujet

\enseignant{E. \textsc{Filliot} \\ R. \textsc{Petit}}  %Nom des enseignants
% Use \\ TO BREAK LINE

\eleves{Hugo \textsc{Callens} \\ Rayan \textsc{Contuliano Bravo} \\ Ethan \textsc{Rogge}}
% \maketitle
\makemargins %Afficher les marges
\makefrontpage
% \maketoc

%=======================================================

% this is the orginal latex code of the template
% \input{example}

%================= Content =============================

\section{Modelisation d'automates en FNC} % (fold)
\label{sec:modelisation_d_automates_en_fnc_}

% Sachant qu'il existe les ensemble $P$ et $N$ qui représentent respectivement 
% les états acceptants et non-acceptants d'un automate, nous avons décidé de 
% modéliser un automate en FNC de la manière suivante :

Voici quelques notations que nous utiliserons dans la suite de ce rapport : 
\begin{itemize}[label=$\bullet$]
    \item $P$ représente l'ensemble des mots acceptés par l'automate 
    \item $N$ représente l'ensemble des mots non-acceptés par l'automate 
    \item $\Sigma$ représente l'alphabet de l'automate 
    \item $k$ représente le nombre au plus d'états de l'automate
\end{itemize}

Afin de modéliser un automate en FNC, il faut tout d'abord définir les variables qui seront utilisées. 
Pour cela, nous avons décidé de créer une variable par état de l'automate, et une variable par transition. 
Ainsi, pour un automate à $n$ états et $m$ transitions, nous aurons $n+m$ variables. 

\subsection{Choix des variables} % (fold)
\label{sub:choix_des_variables}

\subsubsection{Etats} % (fold)
\label{sec:etats}

Nous définissons notre ensemble $Q = \{q_0, q_1, \dots, q_n\}$, qui 
représente l'ensemble des états de l'automate. Sachant que $n \le k$, nous 
pouvons définir les variables $e_i$ comme suit : 
\begin{itemize}[label=$\bullet$]
    \item $q_i = 1$ si l'état $i$ est acceptant 
    \item $q_i = 0$ si l'état $i$ est non-acceptant 
\end{itemize}

% subsubsection Etats (end)

\subsubsection{Transitions} % (fold) 
\label{sec:transitions} 

Nous définissons notre ensemble $T = \{t_1, t_2, \dots, t_m\}$, qui représente 
l'ensemble des transitions de l'automate ou $t_i$ représente le $i$-ème lettre de 
l'alphabet. 
Nous pouvons définir les variables $t_i$ comme suit : 
\begin{itemize}[label=$\bullet$]
    \item $t_i = 1$ si la transition $i$ est utilisée 
    \item $t_i = 0$ si la transition $i$ n'est pas utilisée 
\end{itemize}

\subsection{Contraintes} % (fold)
\label{sub:contraintes}

\begin{itemize}
    \item Il faut qu'il y ait un et un seul état initial 
    \item Il faut qu'il y ait au moins un état final
    \item Il faut que l'ensemble $P$ soit inclus dans l'ensemble des mots acceptés par l'automate 
    \item Un mot se trouvant dans l'ensemble $N$ ne peut pas se trouver dans l'ensemble $P$ 
\end{itemize}
% subsection Contraintes (end)



% subsection Choix des variables (end)



% section Modelisation d'automates en FNC (end)










%================= Bibliography ========================
% \newpage
% \phantomsection % Required if hyperref is used
% \addcontentsline{toc}{section}{References} % Adding bibliography to table of contents
% \printbibliography % Print the bibliography

\end{document}
